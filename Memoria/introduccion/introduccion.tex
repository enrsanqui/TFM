% !TEX root =../pfcTipoETSI.tex
%El anterior comando permite compilar este documento llamando al documento raíz
\chapter{Introducción}\LABCHAP{intro}
%\epigraph{}{}

%\lettrine[lraise=0.7, lines=1, loversize=-0.25]{E}{n}
\lettrine[lraise=-0.1, lines=2, loversize=0.2]{P}{r}oducir burbujas de tamaño micrométrico tiene, hoy día, numerosas aplicaciones que van más allá de aquellas destinadas a procesos  a escala de laboratorio. De hecho, el gran desarrollo y la diversidad de tecnologías desarrolladas en los últimos años han dado lugar a recientes revisiones del estado del arte que tratan con gran detalle tanto los fundamentos que sustentan la producción de microburbujas monodispersas como sus aplicaciones (véase~\cite{Rodriguez-Rodriguez2015b}). Tanto es así que el empleo de microburbujas tiene cabida en procesos de índole tan variada como el tratamiento de aguas, la industria alimentaria, y todo tipo de procesos médicos y farmacológicos, por citar algunos ejemplos. Así, para la obtención de imágenes por ultrasonidos, el uso de microburbujas como angentes de contraste ha demostrado arrojar unos excelentes resultados \cite{Ferrara2007,Kilbanov2006,Postema2005}. Por otro lado, las elevadas necesidades de aireación y el gran porcentaje que esta ocupa en el coste de operación de los biorreactores~\cite{Garcia-Ochoa2009,Rosso2008} hacen que el adecuado control del tamaño y la frecuencia de producción de las burbujas tenga un fuerte impacto en la eficiencia del proceso. 


Cuando se requiere el uso de microburbujas en aplicaciones como las mencionadas anteriormente, se dispone de tres variables que se desean controlar: el diámetro medio de las burbujas, $d_{b}$, la frecuencia de producción de estas, $f_{b}$ y el índice de polidispersión, PDI, ya que para considerar que la producción de microburbujas es monodispera estas tienen que tener un PDI por debajo del 5\%~\cite{Rodriguez-Rodriguez2015b}. Así, en el caso de la oxigenación de biorreactores, típicamente será necesario satisfacer la demanda de oxígeno de los microorganismos presentes, OUR~(\emph{Oxygen Uptake Rate}, de sus siglas en inglés), por lo que el ratio de transferencia de oxígeno~(\emph{Oxygen Transfer Rate}-OTR) puede ser el paso limitante en todo el proceso~\cite{Garcia-Ochoa2009}. El parámetro que controla el OTR es, dado un gradiente de concentración, el coeficiente de transferencia de masa, $k_{L}a$, el cual se encuentra direcamente afectado por la frecuencia y el diámetro de las burbujas. En efecto, el area específica de interfase, i.e. el area por unidad de volumen de las burbujas, depende de forma inversa del diámetro de las burbujas ($a = 6\phi/d_{b}$, con $\phi$ la fracción de gas en el medio). Por lo tanto, a menor diámetro de las burbujas mayor es el coeficiente de transferencia de transferencia de masa, lo que se traduce en última instancia en una reducción del caudal de aire requerido para satisfacer el OUR del cultivo, con el consiguiente ahorro que esta implica. Es por ello que el empleo de difusores de burbuja fina \cite{Sander2017,Rosso2008}, con diámetros 1-3~mm, o incluso de microburbujas \cite{Terasaka2011,Kawahara2009,Sadatomi2005}, con diámetros 10-500~$\mu$m, resulta esencial cuando se trata de aumentar la eficiencia del proceso de aireación; no obstante, también se ha reportado que la presencia de surfactantes y antiespumantes en el medio puede reducir críticamente el valor de dicho coeficiente en los casos en los que se favorece la coalescencia (véase~\cite{Garcia-Ochoa2005}), mientras que la presencia de una determinada concentración de sal (equivalente al lodo presente en las aguas no tratadas) puede inhibir precisamente esta coalescencia~\cite{Sander2017}. 

Como puede observarse, las exigencias de las industrias actuales, que requieren cada vez mayores frecuencias de producción y menores diámetros de las burbujas, conlleva que se hayan desarrollado diferentes tecnologías para poder conseguir una población monodispersa donde se pueda controlar tanto $d_{b}$ y $f_{b}$. Todas estas tecnologías emergentes emplean dispositivos microfluídicos de tamaño milimétrico y submilimétrico que, aunque puedan parecer muy similares entre sí, se fundamentan en principios físicos diferentes que conviene comprender~\cite{Rodriguez-Rodriguez2015b}. De este modo, el capítulo se estructura de la siguiente forma: en primer lugar, se realiza una descripción de las ecuaciones que gobiernan la dinámica de una burbuja que se produce en el seno de un líquido, después, se enumeran las diversas tecnologías que se han desarrollado y que se utilizan para generar una población monodispersa de microburbujas en los diferentes regímenes. En algunas de esta aplicaciones, el papel del gradiente de presión existente en el líquido exterior juega un papel fundamental que se describe con detalle en la \SEC{gradPres}. Finalmente, considerando las pequeñas geometrías empleadas en las tecnologías actuales y el rol determinante del gradiente de presión, se expondrán las ideas que han llevado al desarrollo del dispositivo que en este trabajo se estudia. 

\section{Fundamentos de la generación de burbujas}\LABSEC{fundamentos}

Lejos de lo que pudiera intuirse, la generación de burbujas es un proceso con importantes diferencias respecto al procceso de producción de gotas. En general, para generar gotas de radio $r_{d}$ a una frecuencia $f_{d}\sim U/r_{d}$, basta con injectar el líquido a través de un tubo de radio $r_{t}\sim r_{d}$ a una velocidad $U \gtrsim U_{c}$, con $U_{c} = \left[\sigma/\left(\rho g\right)\right]^{1/2}$ la velocidad capilar\footnote{Se remite al lector interesado en la rotura e inestabilidades de chorros líquidos a la excelente revisión~\cite{Eggers2008}}~\cite{Rodriguez-Rodriguez2015b}. 

Sin embargo, en el caso de la formación  de burbujas para estas condiciones, el comportamiento es diferente. Por un lado, en el caso de generación cuasiestática en el que $U \ll U_{c}$, el radio de la burbuja viene dado por el conocido como \emph{radio de Fritz}, que resulta del balance de esfuerzos de tensión superficial con la fuerza de flotación, $r_{F}/r_{t} \sim \left[3/(2Bo)\right]^{1/3}$, con $Bo = \rho g r_{t}^{2}/\sigma$ el número de Bond, que mide la importancia relativa de los esfuerzos de tensión superficial frente a los esfuerzos de volumen  (gravedad/flotación). Sin embargo, si se desea reducir el diámetro de las burbujas o incrementar la frecuencia aumentando la velocidad de inyección del gas a valores por encima de $U_{c}$, lejos de obtener un chorro de radio comparable al del injector, se obtienen (por encima de cierta velocidad) burbujas de volumen $V_{b} \propto \left(Q_{g}/g^{1/2\right)^{6/5}} > 4/3\pi r_{F}^{3}$~\cite{Rodriguez-Rodriguez2015b}, y lo que es más, si se sigue aumentando la velocidad, el fenómeno de coalescencia puede provocar que las burbujas finalmente obtenidas sean mayores~\cite{Higuera2006}. 

Para explicar estas diferencias entre ambos procesos, si se desprecian los efectos dinámicos del gas (ya que $\rho_{g}/\rho \gg 1$ y $\mu_{g}/\mu  \gg 1$), y se considera que la burbuja es prácticamente esférica, con una presión uniforme en su interior, la dinámica de la burbuja puede describirse a través de la ecuación de Rayleigh-Plesset.

\begin{equation}\LABEQ{rayleighPlesset}
\rho\left(R_{b}\ddot{R}_{b} + \dfrac{3}{2}\dot{R}_{b}^{2}\right) = \Delta p_{exit} - \dfrac{2\sigma}{R_{b}} - 4\mu \dfrac{\dot{R}_{b}}{R_{b}}
\end{equation}

Tal y como se puede observar en~\cite{Rodriguez-Rordiguez2015b}, para que el crecimiento de la burbuja y su posterior colapso tengan lugar, requiere que el término $\Delta p - 2\sigma/R_{b}$ cambie de positivo, valor que adquire en los primeros instantes mientras la burbuja se infla, a negativo, momento en el que las velocidades negativas cerca del inyector  la harán colapsar. La \EQ{rayleighPlesset} junto con la ecuación de continuidad

\begin{equation}\LABEQ{continuidad}
Q_{g} = \dfrac{\mathrm{d}V_{b}}{\mathrm{d}t} \simeq 4\pi R_{b}^{2} \dot{R}_{b}
\end{equation}

y el balance de cantidad de movimiento en la línea de gas

\begin{equation}\LABEQ{continuidad}
p_{0} - p_{exit} = \rho_{g}K\left(Re_{g}\right)\left[Q_{g}/\left(\pi r_{t}^{2}\right)\right]^{2}
\end{equation}

