% !TEX root =../LibroTipoETSI.tex
%El anterior comando permite compilar este documento llamando al documento raíz
\chapter{Notas sobre la Redacción del Texto}\label{chp-01}
\epigraph{Texto elaborado por Marísa Balsa, de la Biblioteca de la Escuela Técnica Superior de Ingeniería de la Universidad de Sevilla}%{Claude Shannon, 1948}

%\lettrine[lraise=0.7, lines=1, loversize=-0.25]{E}{n}
\lettrine[lraise=-0.1, lines=2, loversize=0.2]{S}{e} recomienda consultar la norma UNE 50136:1997: Documentación, presentación de tesis y documentos similares. Por otro lado, existen recursos en línea que ayudan a organizar todo el proceso de elaboración del Trabajo. El personal de la Biblioteca imparte formación y asesoramiento sobre su uso.  
Para establecer la estructura y orden de los datos y partes del Trabajo Fin de Grado nos basamos en la Normativa de los Trabajos Fin de Grado de la Escuela Técnica Superior de Ingeniería de Sevilla (Aprobada en la sesión de la Junta de Escuela de 12 de julio de 2013, modificada en Junta de Escuela de 05 de febrero de 2014) y en la norma UNE 50136:1997: Documentación, presentación de tesis y documentos similares.

\section{Preliminares}
\subsection{Cubierta}
Deberá utilizarse la plantilla aprobada por la Junta de Escuela del 25/4/2014 que se describe en este texto y cuyo modelo puede tomarse de este documento. Debe contener todos los datos que se especifican y en el orden y modelo que se establece: nombre completo del autor, título del Trabajo, nombre completo del tutor, nombre del Departamento, fecha, Grado al que se opta, etc.
\subsection{Portada}
Igulamente deberá utilizarse la plantilla… y contener todos los datos que figuran en ésta, en el orden especificado: el nombre de la Escuela, la titulación y, en su caso, la intensificación, el título del TFG, los nombres del autor, del tutor(es) y, en su caso, del ponente, el Área de Conocimiento, el Departamento y el año de ejecución del proyecto.
\subsection{Resumen} 
Texto que muestra de forma abreviada y precisa el contenido del trabajo. Se podrá también incorporar una versión en inglés que se colocará a continuación precedida del término Abstract.
\subsection{Prefacio o Introducción}
Debe incluir una breve explicación de las razones que han llevado a la realización del Trabajo, el propósito y los objetivos que se pretenden, el ámbito, alcance y límites de la investigación, así como la metodología empleada y, si se considera oportuno, un avance de las conclusiones alcanzadas.  

\subsection{Índice y/o Índice general}
Tabla de contenidos donde se reflejan todas las partes del Trabajo y sus anexos, si los hubiera. Deben aparecer los títulos, en su orden y con indicación de la página en la que se pueden encontrar.
Si el Trabajo consta de varios volúmenes, cada uno deberá llevar su propio índice, pero se debe incluir también un índice general.
\subsection{Lista de ilustraciones y tablas}
Si el Trabajo incluye ilustraciones y tablas se puede añadir un listado que incluya el número identificativo que figura dentro del texto, la leyenda y el número de la página en la que se encuentra.
También es conveniente mencionar los datos sobre las fuentes de donde se han obtenido dichas ilustraciones si no se han incluido en el propio texto de la memoria.
\subsection{Lista de abreviaturas y símbolos}
El Trabajo debe contener las abreviaturas y símbolos internacionalmente reconocidos. Si se incorporan unidades, abreviaturas o acrónimos que puedan ser poco conocidos se deberán explicar brevemente en estas listas.
\subsection{Glosario}
Los términos que requieran definición o explicación se deberán incorporar en un glosario.
\section{Texto principal}
Para su elaboración es importante tener en cuenta su distribución en capítulos y secciones numeradas, siguiendo la plantilla elaborada por…. 
El texto debe comenzar con una introducción que muestre las investigaciones previas existentes sobre el tema y destacar los objetivos y métodos seguidos para llevar a cabo la investigación o análisis del tema tratado.
Para finalizar se deben escribir las conclusiones que deben estar en relación con los objetivos marcados previamente.
\section{Bibliografía}
Debe contener las referencias bibliográficas de los documentos consultados para demostrar las bases del trabajo realizado y avalar los datos incorporados y citados en el texto.
Se elaborará de forma normalizada, para lo que se aconseja utilizar la norma UNE vigente (actualmente la “UNE ISO 690:2013. Información y documentación. Directrices para la redacción de referencias bibliográficas y de citas de recursos de información”).
Para la elaboración de esta parte del Trabajo se recomienda consultar la Web de la Biblioteca de Ingeniería que contiene recursos, guías y ayudas para la elaboración de las referencias bibliográficas. 
\section{Anexos}
Se puede incluir de esta forma material extenso utilizado en el trabajo, importante para justificar los resultados y las conclusiones obtenidas, pero que no es esencial para la comprensión del texto principal. Pueden ser datos estadísticos, legislación, etc.
La paginación debe ser correlativa y continuar la del texto principal. Cada uno de los anexos debe identificarse con una letra mayúscula del alfabeto, comenzando por la letra A, precedida de la palabra Anexo

\endinput
