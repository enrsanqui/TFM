% !TEX root =../pfcTipoETSI.tex
\chapter{Ala acuática bidimensional}\LABCHAP{ala2D}
\pagestyle{esitscCD}
\epigraph{ }{}

%\lettrine[lraise=0.7, lines=1, loversize=-0.25]{E}{l} 
\lettrine[lraise=-0.1, lines=2, loversize=0.25]{A}l final del capítulo anterior se expuso la posibilidad de emplear los gradientes favorables de presión se producen en el borde de ataque de un perfil aerodinámico (cuando sobre el incide una corriente con un ángulo de ataque $\alpha$ determinado) como mecanismo para la formación de microburbujas monodispersas, basándose por analogía en los resultados de~\cite{Evangelio2015b}. El objetivo de este capítulo es pues materializar esta idea diseñando, fabricando y realizando experimentos con un dispositivo generador masivo de microburbujas consistente en un perfil aerodinámico. De este modo, se pretende no sólo demostrar que la producción de burbujas de tamaños milimétricos y submilimétricos no lleva necesariamente aparejado el empleo de dispositivos microfluídicos. Además, se contrastarán las hipótesis realizadas en el capítulo anterior y se explorarán las diferentes dificultades que puedan presentarse en el proceso de escalado. 

La estructura que tomará el capítulo será la siguiente: 
\begin{enumerate}
\item En la \SEC{dessign}, se describe todo lo concerniente al diseño y fabricación del modelo completo, que comprende el dispositivo generador de microburbujas y el montaje experimental del mismo. 
\item En la \SEC{experimentos2D}, se especifican los métodos de experimentación y análisis que serán empleados en la obtención de resultados. De este modo, se describirá la configuración final del \textit{set-up}, el protocolo de experimentación, los datos en bruto obtenidos y el posterior tratamiento para el análisis de los mismos.
\item Finalmente, en  la \SEC{results2D}, se muestran y comentan los resultados obtenidos, realizando las conclusiones pertinentes al final del capítulo
\end{enumerate}

\section{Diseño y fabricación}\LABSEC{dessign}

El objetivo de esta sección es describir de forma detallada los parámetros que configuran el diseño del modelo, tanto aquellos que se refieren al dispositivo en sí como al montaje experimental. Para ello, se describen sucintamente en primer lugar los equipos de los que se dispone y las distintas limitaciones de cada uno de ellos, con el fin de poder acotar el diseño final.

\subsection{Equipos disponibles y limitaciones}\LABSSEC{equipos}



%



%%%%%%%%%%%FIN


\captionsetup[figure]{textformat=period}
\endinput

